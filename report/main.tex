\documentclass[sigconf]{acmart}

\usepackage{booktabs} % For formal tables


\begin{document}
\title{Visualizing User Behavior on the Places and Spaces Website}

\author{Avadhoot Agasti}
\affiliation{School of Informatics and Computing, Bloomington, IN 47408, U.S.A.}
\email{aagasti@indiana.edu}

\author{Sharad Ghule}
\affiliation{School of Informatics and Computing, Bloomington, IN 47408, U.S.A.}
\email{ssghule@iu.edu}

\author{Shreyas Rewagad}
\affiliation{School of Informatics and Computing, Bloomington, IN 47408, U.S.A.}
\email{srewagad@iu.edu}

\author{Leonard Mwangi}
\affiliation{School of Informatics and Computing, Bloomington, IN 47408, U.S.A.}
\email{lmwangi@indiana.edu}


\begin{abstract}

The \textit{Places and Spaces: Mapping Science} exhibit introduces science mapping techniques to the general public and to experts across disciplines for educational, scientific, and practical purposes. The exhibit website \textit{www.scimaps.org} provides information about people behind the exhibit; showcases maps and macroscopes; lists past, present and planned exhibit venues and dates. The website underwent a redesign in 2015 to update the organization and user interface of the website and this study aims to analyze and visualize the changes in behaviour of users of the website due to the redesign. After scraping raw data off of monthly website usage reports in HTML format and cleaning and trasnforming it into usable format, Tableau was used to create an interactive dashboard that would allow a user to gather insights from a number of visualizations. The dashboard allows a user to analyze the data from a high level as well as to drill down into details if required. Several interesting insights that were uncovered using the dashboard are presented.
\end{abstract}


\keywords{Tableau, Visualization, Geospatial, URL, Hits, Pageviews, Parsing, Web Analytics, Line Graph, Tree Map, Dynamic Visualization}

\maketitle

\section{Introduction} \label{intro}

Between the years 2005 and 2014, the \textit{Places and Spaces: Mapping Science} exhibit worked towards the goal of bringing maps of science to the general public. In the year 2015, however, \textit{Places and Spaces} made moves in a direction that marked both a continuation of and a development upon its past achievements. While the exhibit's first decade was mainly devoted to static maps of science, the second decade's mission is devoted to exploring the power and potential of macroscopes; which can be thought of as interactive tools to analyze complex, vast and slow phenomenons in the field of science. The website which acts as a source of information about the exhibit has a visitor base aross the globe and hosts a lot of informational content; videos, games and many science maps. The usage statistics of the website for the period of ten years from 2007 to 2017 are available through Webalyzer reports in HTML format. Visualizing all the information in these HTML files through an interactive dashboard would provide insights about the chnages in web traffic on the website after it was redesigned. Python was chosen as a tool to scrape the data from 120 HTML files and and seggregate it into a set of tables which would then be used to create visualizations. The visualizations include descriptive statistics of visitor demographics, page visits, content downloads; geospatial origin of visits; correlation between exhibit events and user activity.

\section{Client Requirements and Visualization Goals} \label{requirements}
 The intent of the client was to understand how the website is used in order to understand the audience and also to quantify the impact of the website. The requirements from the client helped us direct our analysis and visualizations towards answering below key questions:
\begin{itemize}
 \item What is the geographic origin of the users? \item Are the majority of the users humans or crawlers and search engines?
 \item Is there a correlation between events and web-site traffic?
 \item Are users downloading content, if so, what are they downloading?
 \item What are users searching for? 
 \end{itemize}
Each of the visualizations we created provides insights that enable a user to answer a specific question. We attempted to make each visualization as interactive as possible and visually pleasing while conveying information in the best way possible.

\section{Technical Solution} \label{techsol}

The data pipeline for creating the visualizations has below important steps
as explained in \ref{fig:datapipeline}.

\begin{itemize}
\item Acquire the website usage data:
The scimaps.org uses the Webalizer tool \cite{weblizer}.
Webalizer
analyzes the web server logs to create HTML report which provide various
statistics of web site usage. While the actual web logs are available for
only 2016 the Weblizer HTML reports are available for last 10 years(March
2017 to February 2017). We used these Weblizer HTML reports as source. The
details of Weblizer reports, the exact HTML format and statistics captured is
 explained in the section \ref{sourcedata}.

\item Combine the data in single queriable store:
Since the Weblizer HTML reports does not allow us to query the data, we
required to convert them into a structured format. We decided to convert the
HTML reports into comma separated format (CSV). The section \ref{dataparser}
explains the implementation of the parser program which converts data into
CSV format. While designing the CSV format, we added metadata fields like
year and month so that we can filter the data for a specific duration.

\item Upload the data in visualization tool:
We used Tableau \cite{tableau} as our visualization tool for the
visualizations. Tableau support importing of CSV data.
\item Create individual visualizations:
We created multiple reports in Tableau to satisfy various project
requirements. Each Tableau report tries to answer a group of requirements for
 the project. Each report follow a similar pattern of filtering the data so
 as to maintain consistency across all reports. \ref{viz} section explains
 each visualization in detail.
\item Create a single dashboard by combining all visualizations:
While it is useful to analyze each dataset separately, it also helps to get a
 combined view of the overall website usage. We created dashboard from all
 the visualizations which helps in analyzing all the website usage data in
 one go. The dashboard provides interactive filters using which user can
 slice and dice data and analyze the usage pattern effectively.
\end{itemize}

\begin{figure}[datapipeline]
\centering
\fbox{\includegraphics[width=\linewidth]{img/datapipeline.png}}
\caption{Data Pipeline.}
\label{fig:datapipeline}
\end{figure}


\subsection{Source Data} \label{sourcedata}
\begin{figure}[samplewebanalyzer]
\centering
\fbox{\includegraphics[width=\linewidth]{img/samplewebanalyzerreport.png}}
\caption{Sample Webanalyzer Report.}
\label{fig:samplewebanalyzer}
\end{figure}

As explained in section \ref{techsol}, the Webanalyzer reports in HTML format
are used as source data. These reports are available for last 10 years on
monthly basis. Each report has following sub-sections
 \begin{itemize}
 \item Monthly statistics
 \item Daily statisitcs
 \item Hourly statistics
 \item Top 100 URLs
 \item Top 10 entry pages
 \item Top 10 exit pages
 \item Top 30 referring Sites
 \item Top 20 search strings
 \item Top 15 user agents
 \item Top 10 countries
 \end{itemize}

 Each section in webanalyzer report has HTML table. Figure
 \ref{fig:samplewebanalyzer} explains sample table from webanalyzer HTML
 report.


\subsection{Data Parser} \label{dataparser}
As explained in section \ref{techsol}, each Webanayzer HTML reports is
converted into CSV format. We implemented Data Parser Python script which
scrapes the
Webanalyzer HTML report and converts it into CSV structure. The data parser
uses Python module called BeautifulSoup to parse the HTML. It then iterates
over all 'A' tags to find the section header within HTML report. Finally it
iterates over the HTML table elements consisting TR and TD tags to extract
the data and writes it in CSV file.
 The data parser code is available at \cite{dataparser} repository. The
 figure \ref{fig:daystats} shows sample records from the DAYSTATS.csv which
 is one of the output CSV created by the data parser.

 \begin{figure}[daystats]
\centering
\fbox{\includegraphics[width=\linewidth]{img/dailystats.png}}
\caption{Daily Statistics Sample Records.}
\label{fig:daystats}
\end{figure}

\subsection{Visualizing in Tableau}
Explain how different reports and generated and then integrated in one dashboard





\section{Visualization} \label{viz}
<TODO: Avadhoot: Just introduction of section - 2 liner>


\subsection{Top Countries and Trend} \label{vizcountries}


\begin{figure}
\centering
\fbox{\includegraphics[width=\linewidth]{img/TopCountries1.png}}
\caption{Geospatial Analysis of Web Traffic}
\label{fig:TopCountries}
\end{figure}


\subsection{Top Agents and Trend} \label{viztopagents}
<TODO: Shreyas>


\subsection{Top Referrals} \label{viztoprefs}
Top Referrals data set shows the referring site to <>. To visualize
top referrals, we used bar chart and restricted our data to top 10 due
to shear number of referral. Using bar chart to visualize the data,
allows the users to quickly identify the trend of referrals and helps
to know where to invest based on traffic. 

\begin{figure}
\centering
\fbox{\includegraphics[width=\linewidth]{img/Referral.PNG}}
\caption{Top referrer.}
\label{fig:topreferrer}
\end{figure}

\subsection{Top Searches and Trend} \label{viztopsearches}
We used tree map visualization to plot the top search strings for every
individual years. The tree maps for individual years, specifically 3 years
before the website was reorganized and 1 year after the website was
reorganized are placed side by side. This helps in understanding the trend of
 the search strings. Figure \ref{fig:topsearches} provides the screenshot of
 the visualization. Please refer to the Tableau live implementation to see
 the interactive version of this visualization which showcases many details
 on the mouse-over.

The analysis of this visualization clearly identifies the trend. Before 2016,
 the maximum search strings were related to 'periodic table of elements'
 while in 2016 the focus is shifted towards 'mapping science'. However, these
   two topics are consistently amongst the top 10 searches throughout the
   analysis period.

\begin{figure}
\centering
\fbox{\includegraphics[width=\linewidth]{img/top_search_string_2013_to_2016.png}}
\caption{Top Search Strings in year 2013 to year 2016.}
\label{fig:topsearches}
\end{figure}

\subsection{Most Popular Pages} \label{viztoppages}

TODO: Shreyas

\subsection{Events and Web Site Traffic Corelation} \label{vizevents}

\begin{figure}
\centering
\fbox{\includegraphics[width=\linewidth]{img/EventTraffic.png}}
\caption{Correlation between Exhibit Events and Web Traffic}
\label{fig:EventTraffic}
\end{figure}

In order to answer one of the key questions, we intended to visualize the trends in web traffic when there were exhibit events. We were able to get information about events in past ten years and classified the events as physical exhibits and poster events. The two datasets used for this visualization were combined using inner join in Tableau. There is clearly a correlation between events and web traffic as well as total data acccessed by the users. Physical events tend to affect web traffic by a significant amount and duration of the events plays a role as well.

\subsection{Dashboard} \label{viz_dashboard}
Due to despairing data sources, sometimes it becomes harder to
generate a relationship between datasets, in this case creating
dashboards may seem illogical because the filters would not flow
through to all the visualizations generated. Tableau give us a great
story telling template that allows users to explore such dataset and
make sense of what the data is trying to show. Dashboards can also be
incorporated in the storyline with the same concept of painting the
big picture.





\section{Key Insights} \label{keyinsights}

TODO: Leonard



\section{Other Similar Technologies} \label{similartech}
There are various performance parameters and aspects that aren’t reviewed in the current paper due to unavailability of data. A good web analysis done by mining the user data would yields insights that could boost website traffic and potential business advancements. In this internet age the delivery of such Key Performance Indicators and solutions need to be real-time.  In this section we review some of the tools that could help in web analysis. 

\subsection{Google Analytics}
This is a freeware made by Google to monitor and report website traffic. The tool showcases the descriptive statistics of the website on high-level. We can also procure intricate details and visualize the trend in user behavior pattern across the webpages. Apart from the fine grained details, the tool possesses Google intelligence and Google’s proprietary machine learning library integration. Thus making the Website owner aware of the possible actions that could potentially increase in website traffic. In order to link the tool to the website a tracking code is added to the web pages, this essentially integrates Google analytic in your website. This is also bundled along with other Google services which would help monitor the website via mobile devices.

\subsection{Piwik}
Piwik is a free open-source utility tool that can help the owner analyze the website traffic and determine the performance of various content hosted. Over the years users have added myriad plugins to Piwik to perform in detailed analysis and user profiling of the website traffic. This coupled with mobile integration helps the owner monitor the website traffic.


\section{Conclusions}

To better understand and analyze the ROI for an application, web
traffic analysis is important. Based on the data collected, processed
and analyzed its clear that most of the traffic is generated around
University of Indiana but the traffic as far much reaching to qualify
as global presence. This analysis also helps define where most of the
resources should be invested as the growth of Bot access has seen a
significant growth since 2012 growing to over 30\% in under 5 years
whereas desktop based access has continued to experience consistent
decline of about 20\% for the same duration. Direct site access has
also being dominant over time claiming over 70\% consistently
overtime.


%\end{document}  % This is where a 'short' article might terminate



\appendix
%Appendix A
\section{Work Distribution}
The co-authors of this report worked together on the design of technical
solutions, visualizations, implementation and documentation. Specifially,
below given is the work distribution
\begin{itemize}
\item Avadhoot Agasti
    \begin{itemize}
    \item Team lead and overall coordination.
    \item Data parser implementation.
    \item Visualization of top searches.
    \item Putting together latext template for report writing.
    \item Writing section \ref{techsol} and section \ref{viztopsearches}
     in this
    report.
    \end{itemize}

\item Sharad Ghule
    \begin{itemize}
    \item Visualization of top countries
    \item Visualization of events and website traffic coorelation
    \item Helping with intermediate deliverables
    \item Writing Abstract, section \ref{intro}, section \ref{requirements},
    section \ref{vizevents} and
    section \ref{vizcountries}  in this report.
    \end{itemize}

\item Shreyas Rewagad
    \begin{itemize}
    \item Visualization of top agents
    \item Visualization of most popular URLs
    \item Research on other similar technologies
    \item Writing section \ref{viztopagents}, section \ref{viz_top_urls} and
    section \ref{similartech}  in this report.
    \end{itemize}

\item Leonard Mwangi
    \begin{itemize}
    \item Visualization of top referrals
    \item Visualization of overall statistics
    \item Creating dashboard
    \item Writing section \ref{viztoprefs}, section \ref{viz_dashboard} and
    section \ref{keyinsights} in this report.
    \end{itemize}

\end{itemize}


\section{Acknowledgements}
 The authors thank Prof. Katy Borner for her technical guidance. The
 authors would also like to thank TAs of Information Visualization class for their valued
 support.


\bibliographystyle{ACM-Reference-Format}
\bibliography{main} 

\end{document}
